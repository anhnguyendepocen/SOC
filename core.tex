\documentclass[12pt,letterpaper]{article}
% === graphic packages ===
\usepackage{epsf,graphicx,psfrag}
% === bibliography package ===
\usepackage{natbib}
% === margin and formatting ===
\usepackage{setspace}
%\usepackage{palatino}[1]
%\usepackage[T1]{fontenc}
%\usepackage{palatino}
%\renewcommand{\familydefault}{cmss}
%\usepackage{cmbright}
%\usepackage[T1]{fontenc}
%\usepackage{arev}
%\usepackage[LGR,T1]{fontenc} %% LGR encoding is needed for loading the package gfsneohellenic
%\usepackage[default]{gfsneohellenic}
%\usepackage{times}
\usepackage{fullpage}
\usepackage{color}
\usepackage{endnotes}
% === math packages ===
\usepackage[reqno]{amsmath}
%\usepackage[pdftex]{hyperref}
\usepackage{amsthm}
\usepackage{amssymb,enumerate}
\usepackage[all]{xy}
\usepackage{lscape}
\usepackage[compact]{titlesec}
%\newtheorem{lem} {Lemma}
%\newtheorem{prop}{Proposition}
%\newtheorem{thm}{Theorem}
%\newtheorem{defn}{Definition}
%\newtheorem{cor}{Corollary}
%\newtheorem{obs}{Observation}
 \numberwithin{equation}{section}
% === dcolumn package ===
\usepackage{dcolumn}
\newcolumntype{.}{D{.}{.}{-1}}
\newcolumntype{d}[1]{D{.}{.}{#1}}
% === additional packages ===
\usepackage{url}
\newcommand{\makebrace}[1]{\left\{#1 \right \} }
\newcommand{\determinant}[1]{\left | #1 \right | }



\begin{document}
\begin{center}
\textbf{Social Sciences Inquiry - 3, Spring 2018 (SOSC 13300/1)}\\
Tuesday, Thursday, 930-1050 \\
Gates-Blake Hall 321
\end{center}


\noindent \textbf{Contact Information for Instructors}

\noindent Instructor: Justin Grimmer.\\
Pick Hall 423 \\
Office Hours: Mondays and Wednesdays 3-4pm and by appointment \\
Contact Information: grimmer@uchicago.edu \\

\noindent Teaching Assitant: Esma Ozel\\
Office Hours: See below\\
Contact Information: esmaozel@uchicago.edu\\

\noindent This course provides the final course in the Social Sciences Inquiry in the core.  As this is the final class, the primary task will be to produce a serious piece of social science research. The ultimate goal of this class, then, is to equip you to engage with evidence in the world.  This will be relevant in nearly all facets of your life.  In your job you will be asked to propose remedies for problems and to assess the remedies that other propose. And as a citizen, you will often be asked to assess evidence for a policy.   \\

\noindent When evaluating the world, we can either be pundits or social scientists.  Pundits work backwards from conclusions to find evidence to support their world view.  This class rejects punditry out of hand.  Rather, we will prioritize being social scientists.  We will ask open ended questions about the world and follow evidence to reach a conclusion, however that might contradict our political or social predispositions.  Throughout the class we will attempt to recognize our biases when making these inferences.  \\

\noindent To practice our work as social scientists, we will ask you to produce a serious social science paper that uses what you have learned this year to advance a hypothesis and to test that hypothesis using data.  We will take a general view of what this means. We're open to hypotheses across fields of inquiry and a variety of different types of evidence.  The key is to be rigorous about your evidence.    \\

\noindent The output from this research will be two parts.  You will deliver an in class presentation that succinctly describes your question, hypotheses, findings, and how your evidence leads you to those findings.  You will then deliver a serious piece of social science research that summarizes your research for the quarter.    \\

\noindent In order to understand how to formulate social scientific questions and muster evidence to assess those claims, each meeting this class will take on major public policy controversies. The issues I've chosen for this class are among the most charged in current American political debate.  In class, we will aggressively interrogate positions in the papers, reflect on our political predispositions, and ask what evidence we could find to change our mind.  \\  

\noindent To accomplish this interrogation, we will engage the evidence.  We will assess it's quality, identify its weakness, and compare it to evidence for other conclusions.  In short, we will be putting the \emph{science} into social science.  We will avoid punditry if at all possible.    \\ 

\noindent There are two important things to take note as we examine evidence in this class.  We will ask you to separate out normative claims from empirical ones.  Normative questions are important and it is essential for us to recognize normative commitments at the outset. For example, we might believe that there is a normative reason to grant DACA status to young undocumented immigrants regardless of whatever the empirical literature says about the effects of DACA.  Or, we might construct a normative argument about the right to bear arms that has little to do with the social costs of gun ownership or the effect of guns on crime rates.  \\

\noindent Of course separating normative arguments does not mean they are unrelated to empirical evidence that we might acquire.  The more we know about the consequences of some policy, the better we will be able to answer normative questions.  And normative questions also point us towards the most important empirical questions. \\

\paragraph{A Note on Course Content} A paper's inclusion on this syllabus should not be confused with me endorsing the author, their position, or a tacit announcement about my position on the issue.  As you will see, I've included papers below that I've argued with in print.  Rather, the papers are included here to stoke discussion on how we can use evidence to better inform our policy views. \\ 

\noindent Given the controversial content of what we are discussing, it is essential that discussion in this class is respectful.  We will address the positions people advance, rather than the individual.  



\subsection{Grades}
\noindent The grade breakdown will be as follows:

\begin{itemize}
\item[-] 50\% Participation
\item[-] 10\% Research Proposal
\item[-] 15\% In Class Presentation
\item[-] 25\% Final Paper
\end{itemize}	

\paragraph{Participation}: Students should attend each class session, having read the papers and ready to comment on the content of the papers.  In class, the best students will listen to others, build on the comments offered, and connect their comments to the other arguments being made.  If you are ever worried with your performance in the class on participation, please come talk to me about it, I don't want it to be a mystery!

There will be a biweekly session to work on your paper content that Esma will run.  She will communicate the schedule for this with you.  Attending these sessions contribute to participation grades.  They will also dramatically improve your paper.  


\paragraph{Research Proposal}: On 4/12 you will submit a one-page description of your proposed research project.  In this one page document you will identify your question, why you want to study it, and your hypothesis.  Please discuss the paper with the teaching staff before writing this proposal.  

\paragraph{In Class Presentation}: During the final two weeks of class we will have in class presentations.  Your presentation will be between 8-10 minutes long, with time for a brief Q + A from the class. 

\paragraph{Final Paper}: Your final paper will be an \emph{8-10} page paper that will succinctly describe your research and findings.  The bibliography will not count against the length, but figures and footnotes will.  This paper length aligns with the now widespread move to short papers. 


\section{Preliminary Schedule}

All papers are posted on the course github \url{https://github.com/justingrimmer/SOC} with the file names provided in parentheses.  



\subsubsection{3/27: An Empirical Approach to Controversies in Public Policy}

Question is there a (Liberal/NeoLiberal/Conservative/Corporation) Bias in Academics?
How would we know?

\subsubsection{3/29: Gun Control }
\noindent Matthay, Ellicott C., et al. "In-State and Interstate Associations Between Gun Shows and Firearm Deaths and Injuries." Annals of Internal Medicine 167.12 (2017): 837-844. (MatthayEtAl.pdf) \\


\noindent Lott, Jr, John R., and David B. Mustard. "Crime, deterrence, and right-to-carry concealed handguns." The Journal of Legal Studies 26.1 (1997): 1-68.
(LottMustard.pdf) \\


\noindent Luca, Michael, Deepak Malhotra, and Christopher Poliquin. "Handgun waiting periods reduce gun deaths." Proceedings of the National Academy of Sciences (2017): 201619896. (LucaEtAl.pdf) \\


\subsubsection{4/3--No meeting}

\subsubsection{4/5: Immigration}
\noindent Card, David. "The impact of the Mariel boatlift on the Miami labor market." ILR Review 43.2 (1990): 245-257. (Card.pdf) \\

\noindent Ousey, Graham C., and Charis E. Kubrin. "Immigration and crime: Assessing a contentious issue." Annual Review of Criminology  (2017). (OuseyKubrin.pdf)\\

\noindent Lott, John. "Undocumented Immigrants, US Citizens, and Convicted Criminals in Arizona." (2018). (LottUndocumentedCrime.pdf)\\

\subsubsection{4/10: DACA}
\noindent Kuka, Elira, Na'ama Shenhav, and Kevin Shih. Do Human Capital Decisions Respond to the Returns to Education? Evidence from DACA. No. w24315. National Bureau of Economic Research, 2018. (Kuka.pdf) \\

\noindent Hainmueller, Jens, et al. "Protecting unauthorized immigrant mothers improves their children’s mental health." Science 357.6355 (2017): 1041-1044. (HainmuellerDACAMother.pdf)\\

\noindent Lueders, Hans, Jens Hainmueller, and Duncan Lawrence. "Providing driver’s licenses to unauthorized immigrants in California improves traffic safety." Proceedings of the National Academy of Sciences (2017): 201618991. (HainmuellerLicense.pdf) \\

\subsubsection{4/12: Minimum Wage }

\noindent Card, David, and Alan B. Krueger. "Minimum Wages and Employment: A Case Study of the Fast-Food Industry in New Jersey and Pennsylvania." American Economic Review 84 (1994): 772-793. (CardKrueger.pdf) \\

\noindent Wilson, Mark. "The negative effects of minimum wage laws." (2012). (Wilson.pdf)\\

\subsubsection{4/17: Poverty Alleviation and Negative Income Tax}

\noindent Chase-Lansdale, P. Lindsay, et al. "Mothers' transitions from welfare to work and the well-being of preschoolers and adolescents." Science 299.5612 (2003): 1548-1552. (Chase.pdf)\\

\noindent Moffitt, Robert A. "The negative income tax: would it discourage work." Monthly Lab. Rev. 104 (1981): 23. (Moffit.pdf)\\

\noindent Dahl, Gordon B., and Lance Lochner. "The impact of family income on child achievement: Evidence from the earned income tax credit." American Economic Review 102.5 (2012): 1927-56. (DaleLochner.pdf)\\

\subsubsection{4/19:  Voter Identification Laws} 
\noindent Hopkins, Daniel J., et al. "Voting but for the law: Evidence from Virginia on photo identification requirements." Journal of Empirical Legal Studies 14.1 (2017): 79-128. (HopkinsEtAl.pdf)\\

\noindent Hajnal, Zoltan, Nazita Lajevardi, and Lindsay Nielson. "Voter identification laws and the suppression of minority votes." The Journal of Politics 79.2 (2017): 363-379. (HajnalEtAl.pdf)\\

\noindent Grimmer, Justin, Eitan Hersh, Marc Meredith, Jonathan Mummolo, and Clayton Nall. ``Obstacles  to estimating voter ID laws' effect on turnout" The Journal of Politics. 80 (3) (GrimmerFinal.pdf)\\

\noindent Hajnal, Zoltan, John Kuk, and Nazita Lajevardi. ``We All Agree: Strict Voter ID Laws Disproportionately Burden Minorities". 2018. 80 (3). (To Be Posted) \\

\noindent Grimmer et al.  Memo: The Voter ID paper controversy at the JOP.  (To Be Posted)\\

\subsubsection{4/24: Gerrymandering} 
\noindent Rodden, Jonathan and Jowei Chen. 2013. ``Unintentional Gerrymandering: Political Geography and Electoral Bias in Legislatures", 2013, Quarterly
Journal of Political Science 8: 239-269 (ChenRodden.pdf) \\

\noindent McCarty, Nolan, Keith T. Poole, and Howard Rosenthal. "Does gerrymandering cause polarization?." American Journal of Political Science 53.3 (2009): 666-680. (McCartyPooleRosenthal.pdf)\\

\subsubsection{4/26: Lobbying and Campaign Donations}
\noindent Blanes i Vidal, Jordi, Mirko Draca, and Christian Fons-Rosen. "Revolving door lobbyists." American Economic Review 102.7 (2012): 3731-48. (BlanesRevolving.pdf)\\

\noindent Ansolabehere, Stephen, John M. De Figueiredo, and James M. Snyder Jr. "Why is there so little money in US politics?." Journal of Economic perspectives 17.1 (2003): 105-130. (SoLittleMoney.pdf)\\

\noindent Hall, Richard L., and Alan V. Deardorff. "Lobbying as legislative subsidy." American Political Science Review 100.1 (2006): 69-84. (LegislativeSubsidy.pdf)\\


\subsubsection{5/1: No Meeting, Work on Papers } 

\subsubsection{5/3: No Meeting, Work on Papers } 

\subsubsection{5/8: Term Limits} 
\noindent Fouirnaies, Alexander and Andrew B. Hall.2018.``How Do Electoral Incentives
Affect Legislator Behavior?" Stanford University Mimeo.  (Fouirnaies\_Hall\_Electoral\_Incentives.pdf)\\

\subsubsection{5/10:  Police Tactics } 
\noindent Mummolo, Jonathan. "Modern Police Tactics, Police-Citizen Interactions, and the Prospects for Reform." The Journal of Politics 80.1 (2018): 1-15. (MummoloSQF.pdf)\\
)

\noindent Weaver, Vesla M., and Amy E. Lerman. "Political consequences of the carceral state." American Political Science Review 104.4 (2010): 817-833. (WeaverLerman.pdf)\\

\subsubsection{5/15: New Jim Crow and the War on Drugs}
\noindent Alexander, Michelle. "The New Jim Crow." Ohio St. J. Crim. L. 9 (2011): 7. (NewJimCrow.pdf)\\

\noindent Aviram, Hadar, Allyson Bragg, and Chelsea Lewis. "Felon Disenfranchisement." Annual Review of Law and Social Science 13 (2017): 295-311. (FelonDisenfranchisement.pdf)\\
\subsubsection{5/17: Student Presentations} 

\subsubsection{5/22: Student Presentations} 

\subsubsection{5/24: Student Presentations} 

\subsubsection{5/29: Student Presentations}





\end{document}
